\section{T\'erminos y condiciones del acuerdo}
\subsection{Partes involucradas}

Este acuerdo se establece entre la Direcci\'on de Servicios de
Informaci\'on y Tecnolog\'ia y la Facultad de Ciencias de la
Universidad de los Andes.

\subsection{Contactos}
\label{contactos}
\begin{itemize}
\item Juan Pablo Mallarino, Facultad de Ciencias, Ext., 
\item Andr\'es Holgu\'in, Coordinador Administrativo, Ext., 
\end{itemize}

\subsection{Descripci\'on del acuerdo}
Este acuerdo est\'a relacionado con la provisi\'on del servicio descrito en el Cap\'itulo XX. El acuerdo incluye las condiciones de prestaci\'on del servicio, su alcance y los compromisos entre las parts.

\subsection{Fechas de Vigencia}

{\bf Inicio:}\\

{\bf Fin:}\\


\subsection{Alcance del acuerdo}
En el acuerdo se incluyen las condiciones generales de la prestaci\'on
del servicio descrito en la secci\'on XXX> Se excluyen de este acuero
elementos de infraestructura de informaci\'on de los cuales la DSIT no
tiene control tales como: estaciones cliente. 

\subsection{Revisi\'on del acuerdo}
Cualquiera de los contactos de la secci\'on \ref{contactos} podr\'a
solicitar la revisi\'on del acuerdo de servicio en el momento que lo
considere necesario con el fin de evaluar la calidad del servicio y el
cumplimiento de los compromiso, para luego formalizar los ajustes a
que haya lugar. El presente acuerdo se mantiene vigente mientras no haya
solicitud de revisi\'on. El servicio descrito por el acuerdo puede
incorporar revisiones si ambas partes aceptan los cambios propuestos.
 
{\bf \'Ultima revisi\'on}: \\

{\bf Pr\'oxima revisi\'on}: (trimestre posterior a la firma).

\subsection{Definici\'on de responsabilidades}

\subsubsection{Responsabilidades de la DSIT}
Las siguientes son las responsabilidades de la DSIT:
\begin{itemize}
\item Alojar en su infraestructura de servidores (Data Center) los
  servidores de \icar. La responsabilidad de alojamiento incluye las
  siguiente labores de mantenimiento y administraci\'on:
  \begin{itemize}
  \item Sistema Operativo y Aplicaciones
  \item Seguridad
  \item Monitoreo
  \item Backup
  \item Informaci\'on y seguimiento
  \item Documentaci\'on para el usuario
  \item Cuentas de Usuario
  \item Adminitraci\'on de recursos computacionales
  \item Generaci\'on de procedimientos.
  \end{itemize}
\item Ofrecer un punto \'unico de contacto, a trav\'es del coordinador
  administrativo (Secci\'on \ref{contactos}), para que los usuarios
  del servicio reporten sus solicitudes e incidentes y puedan
  consultar acerca de su avance.
\end{itemize}

\subsubsection{Responsabilidades del cliente}

\begin{itemize}
  \item Comunicaci\'on
\item Administraci\'on de usuarios
\item Generaci\'on de solicitudes
\item Soporte de aplicaciones
\item Administraci\'on de recursos computacionales
\item Informar cualquier incidente o requerimiento a trav\'es de
  nuestro \'unico punto de contacto xxxx o a trav\'es de la
  extensi\'on xxxx enl os horarios que se describen en la Secci\'on
  \ref{horarios}. 
\end{itemize}

\subsubsection{Responsabilidades del usuario}
Las siguientes son las responsabilidades del usuario.

\begin{itemize}
\item Comunicaci\'on
\item Gesti\'on de "Jobs"
\item Acatar las "Normas de uso de tecnolog\'ia"
\end{itemize}

